%
% !TEX root = ../main.tex
%
	
\chapter{}
\label{sec:appendix}

\section{Explicit matrix inverse}
\label{subsec:matrixinverse}

Here we provide the explicit expression for the inverse of matrix in \cref{eq:numericalscheme},
%%%
\begin{equation*}
  \begin{bmatrix}
	\bm{I} 	&	0		&	0 \\
	0	& 	\bm{M}(\bm{q}) 	&	-\frac{\partial\bm{\phi}}{\partial\bm{q}}^{\intercal} \\
	0 	& -\frac{\partial\bm{\phi}}{\partial\bm{q}} & 0\\
  \end{bmatrix}^{-1}
  = 
  \begin{bmatrix}
	\bm{I} 	&	0		&	0 			\\
	0	& 	\bm{X}_{11} &	\bm{X}_{12} \\
	0 	& 	\bm{X}_{21} & 	\bm{X}_{22}
  \end{bmatrix},
\end{equation*}
%%%
where the quantities $\bm{X}_{11}\in\mathbb{R}^{n\times n}$,
$\bm{X}_{12}\in\mathbb{R}^{n\times m}$, $\bm{X}_{21}\in\mathbb{R}^{m\times n}$,
$\bm{X}_{22}\in\mathbb{R}^{m\times m}$ are defined as follows,
%%%
\begin{subequations}
  \begin{align*}
    \bm{X}_{11} &= 
    \bm{M}(\bm{q})^{-1}- \bm{M}(\bm{q})^{-1}
    \frac{\partial\bm{\phi}}{\partial\bm{q}}^{\intercal}
    \left(\frac{\partial\bm{\phi}}{\partial\bm{q}}
    \bm{M}(\bm{q})^{-1}\frac{\partial\bm{\phi}}{\partial\bm{q}}^{\intercal}\right)^{-1}
    \frac{\partial\bm{\phi}}{\partial\bm{q}}\bm{M}(\bm{q})^{-1}, 
    \\
	\bm{X}_{12} &=
	-\bm{M}(\bm{q})^{-1}\frac{\partial\bm{\phi}}{\partial\bm{q}}^{\intercal}
	\left(\frac{\partial\bm{\phi}}{\partial\bm{q}}\bm{M}(\bm{q})^{-1}
	\frac{\partial\bm{\phi}}{\partial\bm{q}}^{\intercal}\right)^{-1},
	\\
    \bm{X}_{21} &=
    - \left(\frac{\partial\bm{\phi}}{\partial\bm{q}}
    \bm{M}(\bm{q})^{-1}\frac{\partial\bm{\phi}}{\partial\bm{q}}^{\intercal}\right)^{-1}
    \frac{\partial\bm{\phi}}{\partial\bm{q}}\bm{M}(\bm{q})^{-1},
    \\
    \bm{X}_{22} &=
    -\left(\frac{\partial\bm{\phi}}{\partial\bm{q}}
    \bm{M}(\bm{q})^{-1}\frac{\partial\bm{\phi}}{\partial\bm{q}}^{\intercal}\right)^{-1}.
  \end{align*}
\end{subequations}
%%%
Notice that as expected $\bm{X}_{12} = \bm{X}_{21}^{\intercal}$,
moreover in the inversion formula appears the expression,
%%%
\begin{equation*}
  \left(\frac{\partial\bm{\phi}}{\partial\bm{q}}\bm{M}(\bm{q})^{-1}
  \frac{\partial\bm{\phi}}{\partial\bm{q}}^{\intercal}\right)^{-1}
\end{equation*}
%%%
which is exactly the one in \cref{eq:compression} for which we must
guarantee the invertibility.
Notice also that in the case of a single constraint,
i.e., $\phi(\bm{q})\in\mathbb{R}$ the following expression can be simplified as follows,
%%%
\begin{equation*}
  \bm{M(\bm{q})}^{-1}\frac{\partial\bm{\phi}}{\partial\bm{q}}^{\intercal}
  \left(\frac{\partial\bm{\phi}}{\partial\bm{q}}\bm{M}(\bm{q})^{-1}
  \frac{\partial\bm{\phi}}{\partial\bm{q}}^{\intercal}\right)^{-1}
  \frac{\partial\bm{\phi}}{\partial\bm{q}}\bm{M}(\bm{q})^{-1}
  = \frac{\bm{M(\bm{q})}^{-1}\frac{\partial\bm{\phi}}{\partial\bm{q}}^{\intercal}
  \frac{\partial\bm{\phi}}{\partial\bm{q}}\bm{M(\bm{q})}^{-1}}
  {\frac{\partial\bm{\phi}}{\partial\bm{q}}\bm{M}(\bm{q})^{-1}
  \frac{\partial\bm{\phi}}{\partial\bm{q}}^{\intercal}}.
\end{equation*}
