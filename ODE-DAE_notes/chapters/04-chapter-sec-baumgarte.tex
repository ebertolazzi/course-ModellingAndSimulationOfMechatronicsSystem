%
% !TEX root = ../main.tex
%

\section{Baumgarte's stabilization}

A very popular way to stabilize the constraint $\bm{\phi}(\bm{q},t)=0$
and thus avoid the drift effect is using Baumgarte's stabilization,
see \cite{baumgarte1972stabilization}.
This idea is well presented with other techniques in the survey paper
\cite{ascher1995stabilization}.
The key idea is to substitute the constraint with a \emph{differential equation}
that \emph{asymptotically} satisfies $\bm{\phi}(\bm{q},t)=0$.
Obviously this technique is effective only if you are interested in
the \emph{long term} behaviour, since during transients the constraint
may be violated. Since constrained mechanical systems require to
differentiations to eliminate the Lagrange multipliers,
we consider a linear second order stable differential equation of the form,
%%% 
\begin{equation}
	\label{eq:secondordersystem}
	\frac{\diff^2 z(t)}{\diff t^2} +
	2\xi\omega_{\textnormal{n}}\frac{\diff z(t)}{\diff t}
	+ \omega_{\textnormal{n}}^2z(t) = 0,
\end{equation}
%%%
where $z(t)\in\mathbb{R}$. In this form we recognize the well known
equation for second order linear mechanical or electrical systems,
such as the \emph{RLC circuit} or the \emph{mass spring damper}.
In these framework $\xi\in\mathbb{R}_{>0}$ is the \emph{damping ratio}
and $\omega_{\textnormal{n}}\in\mathbb{R}_{>0}$ is the \emph{natural frequency}.
Depending on the value of $\xi$ different cases can arise:
%%%
\begin{itemize}
	\item $0<\xi<1$: solution $z$ is \emph{underdamped},
	\item $\xi=1$: solution $z$ is \emph{critically damped},
	\item $\xi>1$: solution $z$ is \emph{overdamped}.
\end{itemize}
%%%
The general form for the solution is the following
%%%
\begin{equation*}
	z(t) = k_{1}e^{\left(-\xi+\sqrt{\xi^2-1}\right)\omega_{\textnormal{n}}t}
	+ k_{2}e^{\left(-\xi-\sqrt{\xi^2-1}\right)\omega_{\textnormal{n}}t}, 
\end{equation*}
%%%
where $k_{1},k_{2}\in\mathbb{R}$ are constants determined by the initial
conditions $z_{0} = z(t=0)$ and $\dot{z}_{0} = \dot{z}(t=0)$.
Solutions for the different cases can be arranged in different ways,
here we show some of them. For a deeper treatment see for example
\cite{rao1995mechanical}.
Let us start from the underdamped case; solution takes the form
%%%
\begin{equation}
	\label{eq:underdamped}
	z(t) = c_ {1}e^{-\xi\omega_{\textnormal{n}}t}
	\cos\left({\sqrt{1-\xi^2}\omega_{\textnormal{n}}t-c_{2}}\right),
\end{equation}
%%%
where constants $c_{1},c_{2}\in\mathbb{R}$ are given by,
%%% 
\begin{equation*}
  c_{1} = \frac{\sqrt{z_{0}^2
  \omega_{\textnormal{n}}^2+\dot{z}_{0}^2+2z_{0}\dot{z}_{0}\xi
  \omega_{\textnormal{n}}}}{\sqrt{1-\xi^2}\omega_{\textnormal{n}}}
  \qquad
   c_{2} = \mathrm{arctan}\left(\frac{\dot{z}_{0}+\xi\omega_{\textnormal{n}}z_{0}}
  { x_ {0}\omega_{\textnormal{n}}\sqrt{1-\xi^2}}\right).
\end{equation*}
%%%
As we can observe in \cref{eq:underdamped} solution is a cosine
function modulated by a decreasing exponential function.
For the critically damped case a linear term appears and the solution takes the form, 
%%%
\begin{equation}
	\label{eq:criticallydamped}
	z(t) = (c_{1}+c_{2}t)e^{-\omega_{\textnormal{n}}t},
\end{equation}
%%%
where $c_{1},c_{2}\in\mathbb{R}$ are given by $c_{1} = z_{0}$ and 
$c_{2} = \dot{z}_{0}+\omega_{\textnormal{n}}z_{0}$.
Finally for the overdamped case the solution takes the form
%%%
\begin{equation}
	\label{eq:overdamped}
	z(t) = c_{1}e^{\left(-\xi+\sqrt{\xi^2-1}\right)\omega_{\textnormal{n}}t} 
	+ c_{2}e^{\left(-\xi-\sqrt{\xi^2-1}\right)\omega_{\textnormal{n}}t}, 
\end{equation}
%%%
with constants, 
%%%
\begin{equation*}
  c_{1} = \frac{x_{0}\omega_{\textnormal{n}}
  \left(\xi+\sqrt{\xi^2-1}\right)+\dot{x}_{0}}{2\omega_{\textnormal{n}}\sqrt{\xi^2-1}},\qquad
  c_{2} = \frac{-x_{0}\omega_{\textnormal{n}}
  \left(\xi-\sqrt{\xi^2-1}\right)-\dot{x}_{0}}{2\omega_{\textnormal{n}}\sqrt{\xi^2-1}}.
\end{equation*}
%%%
However, regardless of the specific form of the solution,
all the three cases satisfy a \emph{global exponential stability}
property, i.e., $\forall(z_{0},\dot{z}_{0})\in\mathbb{R}\times\mathbb{R}$
the solution $z(t,z_{0},\dot{z}_{0})$ satisfies
%%%
\begin{equation*}
	\lim_{t\rightarrow\infty} z(t,z_{0},\dot{z}_{0}) = 0.
\end{equation*}
%%%
The notation $z(t,z_{0},\dot{z}_{0})$ is just to stress that the
solution depends on time but also from initial conditions $(z_{0},\dot{z}_{0})$.
This consideration suggests to substitute the constraint
$\bm{\phi}(\bm{q},t)$ with an ODE of the form (\ref{eq:secondordersystem}).
In this way, despite the initial conditions, the solution will asymptotically
satisfies $\bm{\phi}(\bm{q},t)=0$.
Thus the explicit \emph{Baumgarte's stabilization} for DAEs of index three takes the form,	
%%%
\begin{equation*}
	\frac{\diff^2\bm{\phi}}{\diff t^2} 
	+ 2\xi\omega_{\textnormal{n}}\frac{\diff\bm{\phi}}{\diff t}
	+ \omega_{\textnormal{n}}^2\bm{\phi} = 0.
\end{equation*}
%%%
Computing explicitly the terms for mechanical systems the stabilization results,
%%%
\begin{equation}
	\frac{\diff}{\diff t}\frac{\partial\bm{\phi}}
	     {\partial\bm{q}}\dot{\bm{q}}+\frac{\partial\bm{\phi}}
	     {\partial\bm{q}}\ddot{\bm{q}}+\frac{\partial^{2}\bm{\phi}}
	     {\partial t^2} + 2\xi\omega_{\textnormal{n}}
	     \frac{\partial\phi}{\partial\bm{q}}\dot{\bm{q}} 
	+ 2\xi\omega_{\textnormal{n}}\frac{\partial\bm{\phi}}{\partial t}+\omega_{\textnormal{n}}^2\bm{\phi} = 0.
\end{equation}
%%%
which can be re-arranged in the following form
%%%
\begin{equation}
	\label{eq:baumstability}
	-\frac{\partial\bm{\phi}}{\partial\bm{q}}\dot{\bm{p}}
	= \frac{\diff}{\diff t}\frac{\partial\bm{\phi}}{\partial\bm{q}}\bm{p}
	+\frac{\partial^{2}\bm{\phi}}{\partial t^2}
	+ 2\xi\omega_{\textnormal{n}}\frac{\partial\phi}{\partial\bm{q}}\bm{p}
	+ 2\xi\omega_{\textnormal{n}}\frac{\partial\bm{\phi}}{\partial t}
	+\omega_{\textnormal{n}}^2\bm{\phi} :=\bm{m}(\bm{q},\bm{p},t).
\end{equation}
%%%
Now \cref{eq:baumstability} can be easy implemented without any modification 
in \cref{eq:numericalscheme}; indeed it is sufficient to redefine $\bm{m}(\bm{q},\bm{p},t)$
as in \cref{eq:baumstability}.
However a question is still open, how can we tune the parameters 
$\omega_{\textnormal{n}}$ and $\xi$?. Unfortunately there is not general rule,
but it is useful to notice that the velocity of convergence to zero of the solution
depends on both$\xi$ and $\omega_{\textnormal{n}}$.
Therefore it is a good idea to select them in order to obtain an high
decay rate compatibly with the \emph{numerical stability}.
This means that the dynamics associated to the constraint $\bm{\phi}(\bm{q},t)$
cannot be too fast compared to other ones, otherwise the system could be very
hard (stiff) to simulate numerically.
Indeed systems where two or more very different time scales are involved are
usually called \emph{stiff problems} or \emph{stiff systems}.
These may cause some numerical problems, since stiff equations require
an integration step extremely small and are numerically unstable.
In general it is hard to provide a unique definition of stiffness,
but the main idea is that the differential equations includes some
terms that can lead to rapid variation in the solution.
%%%
\begin{example}
	Continuing \cref{ex:pendulum2} we want use the Baumgarte's
	technique to stabilize the pendulum. Therefore the idea is to substitute the constraint
	%%%
	\begin{equation*}
		2\bm{p}^{\intercal}\bm{p}+2\bm{q}\dot{\bm{p}} = 0
	\end{equation*}
	%%%
	with
	%%%
	\begin{equation*}
		-2\transp{\bm{q}}\dot{\bm{p}} = 2\transp{\bm{p}}\bm{p}
		+ 4\omega_{\textnormal{n}}\xi\transp{\bm{q}}\bm{p}
		+ \omega_{\textnormal{n}}^{2}(\transp{\bm{q}}\bm{q}-\ell^{2}).
	\end{equation*}
	%%%
	Thus final equations appears as, 
	%%%
	\begin{equation}
		\label{eq:examplewithbaumga}
		\begin{bmatrix}
			\bm{I}_{2} 	& 0 & 0 \\
			0 			& m\bm{I}_{2} & -2\bm{q} \\ 
			0 			& -2\transp{\bm{q}}  & 0
		\end{bmatrix}
		\begin{bmatrix}
			\dot{\bm{q}} \\
			\dot{\bm{p}} \\
			\lambda
		\end{bmatrix}
		= 
		\begin{bmatrix}
			\bm{p} \\
			-mg\bm{e}_{2}\\
			2\transp{\bm{p}}\bm{p} + 4\omega_{\textnormal{n}}\xi\transp{\bm{q}}\bm{p} 
			+ \omega_{\textnormal{n}}^{2}(\transp{\bm{q}}\bm{q}-\ell^{2})
		\end{bmatrix}.
	\end{equation}
	%%%
	Again to appreciate the effects of Baumgarte's stabilization we can simulate
	\cref{eq:examplewithbaumga} with same parameters as \cref{ex:pendulum2}
	and stabilization parameters $\xi = 0.8$,
	$\omega_{\textnormal{n}} = 2$.
	The result is showed in \cref{fig:trackingerros}.
	%%%
	\begin{figure}[htbp]
		\centering
		\begin{tikzpicture}
			\begin{axis}
				[xmin=-3,xmax=3,ymin=-3,ymax=3,grid=both,width=8cm,height=8cm]
				\draw (axis cs:0,0) circle [black, radius=2.5];
				\addplot[dotted,samples=10,color=red]table[x=x1,y=y1]{data/stab.txt};
			\end{axis}
		\end{tikzpicture}
		\caption{Pendulum simulation with Baumgarte's stabilization.}
		\label{fig:trackingerros}
	\end{figure}
	%%%
\end{example}

\endinput
		
		