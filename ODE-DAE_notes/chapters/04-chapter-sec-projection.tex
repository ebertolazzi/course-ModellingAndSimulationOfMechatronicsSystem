%
% !TEX root = ../main.tex
%
\section{Projection method}
%%%
Another way to stabilize the constraint $\bm{\phi}(\bm{q},t)=0$ and avoid the drift
effect is using the so-called Projection method.
The idea is to firstly reduce the index of the DAE, thus obtaining an ODE
and the aforementioned \emph{hidden constraints}.
Then the resulting ODE can be numerically integrated, leading to a generic
intermediate solution $\widetilde{\bm{q}}(t_n)$.
This temporary solution is projected to the hidden constraints in order to
minimise the error between the numerical integration solution and the actual
system constraints, which is a \textit{nonlinear constrained least squares problem}
because of the norm chosen.
The projection gives the orthogonal projection to the contraints.

It should be noticed that unlike the previously presented Baumgarte's stabilization
method this technique is useful and particularly effective if you are interested
in the \emph{short term} behaviour, since even during transients the constraint
is always satisfied.
Different information from the original and reduced system can be used for
various projection methods. For example, we can decide to get an advantage
by first projecting the ODE solution $\widetilde{\bm{q}}(t_n)$ to the position
constraints and than to the velocity constraints.

\begin{example}
	%%%
	\begin{figure}[htbp]
		\centering
		\begin{tikzpicture}
			\begin{axis}
				[xmin=-3,xmax=3,ymin=-3,ymax=3,grid=both,width=8cm,height=8cm]
				\draw (axis cs:0,0) circle [black, radius=2.5];
				\addplot[dotted,samples=10,color=red]table[x=x1,y=y1]{data/stab.txt};
			\end{axis}
		\end{tikzpicture}
		\caption{Pendulum simulation with Projection method.}
		\label{}
	\end{figure}
	%%%
	Continuing \cref{ex:pendulum2} we want use the 
	Projection technique to stabilize the pendulum.
	The numerical scheme for integrating the reduced DAE
	is that in \cref{eq:numericalschemependulum}.
	We want now to project the results of the generic ODE
	numerical integrator to the following \emph{hidden constraints},
	%%%
	\begin{equation*}
		\begin{split}
		\dfrac{\diff\phi}{\diff t} &= 2\transp{\bm{q}}\bm{p} = 0.\\[1em]
		\dfrac{\diff^{2}\phi}{\diff t^2} & = 2\transp{\dot{\bm{q}}}\bm{p} 
		+ 2\transp{\bm{q}}\dot{\bm{p}} = 2\transp{\bm{p}}\bm{p} 
		+ 2\transp{\bm{q}}\dot{\bm{p}} = 0.
		\end{split}
	\end{equation*}
	%%%
	At each time step, the resulting orthogonal projection to the constraints is equal to
	%%%
	\begin{equation}
		\begin{split}
		\text{minimize:} \quad & \dfrac{1}{2}|| \widetilde{\bm{q}} - \bm{q} ||_2
		\\[1em]
		\text{subject to:} \quad & \bm{\phi}(\bm{q},t) = \transp{\bm{q}}\bm{q} - \ell^{2} = 0
		\end{split}
	\end{equation}
	%%%
	for the position coordinates projection, and
	%%%
	\begin{equation}
		\begin{split}
		\textrm{minimize:} \quad &
		\dfrac{1}{2}|| \widetilde{\bm{p}} - \bm{p} ||_2\\[1em]
		\text{subject to:} \quad & \frac{\diff\bm{\phi}}{\mathrm{d}\bm{q}}(\bm{q},t)\bm{p}
		+ \frac{\diff\bm{\phi}}{\mathrm{d}t}(\bm{q},t)= 0
		\end{split}
	\end{equation}
	%%%
	for the velocity coordinates projection.
\end{example}

\endinput
