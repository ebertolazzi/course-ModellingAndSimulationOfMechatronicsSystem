%
% !TEX root = ../main.tex
%

\chapter{Linear DAE}
	\label{ch:linearDAE}
	\section{Linear Differential Algebraic Equations}
		In this chapter linear implicit differential equations or DAE s of the form
		\textbf{}
		\begin{equation}
			\label{eq:daelinear}
			\bm{E}\dot{\bm{x}}(t) = \bm{A}\bm{x}(t)+\bm{f}(t)
		\end{equation}
		are studied. In the simplest case, $\bm{E},\bm{A}\in\mathbb{R}^{n\times n}$ are square constant matrices and $\bm{f}(t):\mathbb{R}\rightarrow\mathbb{R}^{n}$ is some external perturbation independent of $\bm{x}(t)$. Such equations occur for example by linearization of autonomous non-linear problems with respect to constant (or critical) solutions, here $\bm{f}(t)$ is a perturbation that pushes away the solution from the nominal trajectory. Other cases are modelling of  linear electrical circuits, simple mechanical systems or, in general, linear systems with additional linear algebraic  constraints.
		The first question that we need to answer, regards existence and uniqueness of a solution $\bm{x}(t):\mathbb{R}\rightarrow\mathbb{R}^{n}$.
		Of course the only interesting case is when $\bm{E}$ is singular (not invertible). Otherwise it is sufficient to invert $\bm{E}$ and the DAE s can be written as a ODE s as follows,
		\begin{equation}
			\label{eq:odelinear}
			\dot{\bm{x}}(t) = \bm{E}^{-1}\bm{A}\bm{x}(t)+\bm{E}^{-1}\bm{f}(t).
		\end{equation}
		For \cref{eq:odelinear} it is sufficient to assume $\bm{f}(t)$ piecewise continuous \footnote{Weaker conditions for $\bm{f}(t)$ can be used, for example it is sufficient that $\bm{f}(t)$ is Lebesque measurable.} and existence and uniqueness are ensured, see for example \cite{coddington1955theory}.  
		Conversely when matrix $\bm{E}$ is singular, it is not a priori clear if the solution exists or not. A singular matrix $\bm{E}$ means that ODE s are mixed with linear algebraic constraints. At this point is important consider some regularity conditions for the pair $(\bm{E},\bm{A})$, as shown in the following example. 
		\begin{example}
			Consider the DAE s with $\bm{x}(t)=(x_{1}(t);x_{2}(t))\in\mathbb{R}^{2}$, $\bm{f}(t)=0$,$\forall t\in[0,\infty)$ and matrices $\bm{E},\bm{A}$ given by 
			\begin{equation}
				 \bm{E} = 
				 \begin{bmatrix}
				 	1 & 0 \\
				 	0 & 0
				 \end{bmatrix}, \quad
				 \bm{A} = 
				 \begin{bmatrix}
				 	0 & 1 \\
				 	0 & 0
				 \end{bmatrix}.
			\end{equation}
			This DAE s corresponds to the scalar differential equation $\dot{x}_{1}(t) = x_{2}(t)$. Fixing an initial condition $\bm{x}_{0}=(0;0)\in\mathbb{R}^{2}$ a solution is provided by
			\begin{equation}
				\bm{x}(t) = \int_{0}^{t}
				\begin{bmatrix}
					x_{2}(\tau) \\
					\alpha(\tau)
				\end{bmatrix}
				\diff\tau,
			\end{equation}
			where $\alpha(\tau)$ represents an arbitrary function that satisfies the initial condition.  For example let us consider $\alpha(t) = \sin(t)$ which is compatible with the initial condition ($\sin(0)=0$), then the solution $\bm{x}(t)$ is
			\begin{equation}
				\bm{x}(t) = 
				\begin{bmatrix}
					-\sin(t) \\
					-\cos(t)
				\end{bmatrix}.
			\end{equation}	
			Nevertheless other choices are possible, i.e. $\alpha(t) = t$, then the solution is
			\begin{equation}
				\bm{x}(t) = 
				\begin{bmatrix}
					t^{3}/6\\
					t^{2}/2
				\end{bmatrix}.
			\end{equation}	
			It is therefore clear that the solution is not unique and that we must assume some regularity assumptions for the pair $(\bm{E},\bm{A})$.
		\end{example}
		In order to clarify the existence and uniqueness of a solution for \cref{eq:daelinear} we need the following definition.
		\begin{definition}
			Matrix pair $(\bm{E},\bm{A})$ is said to be a \emph{regular pencil} if there exist a scalar $\lambda_{0}\in\mathbb{R}$, such that
			\begin{equation}
				\determinant{\bm{A}-\lambda_{0}\bm{E}} \neq 0.
			\end{equation}
		\end{definition}
		\begin{example}
			The pair of matrices $\bm{E},\bm{A}\in\mathbb{R}^{3\times 3}$ defined below is a non regular (singular) pencil.
			\begin{equation}
				\bm{E} = 
				\begin{bmatrix}
					1 	& 0 	& 0 \\
					-2 	& 0 	& 0 \\
					0 	& -2 	& 0 
				\end{bmatrix},
				\quad
				\bm{A} = 
				\begin{bmatrix}
					1 & 0 & 0 \\
					0 & 1 & 0 \\
					0 & 0 & 0
				\end{bmatrix}
			\end{equation}
			Since computing the determinant of the pencil is identically zero for every value of $\lambda_{0}\in\mathbb{R}$. 
			\begin{equation}
				\textnormal{det}\left(\bm{A}-\lambda_{0}\bm{E}\right) =\textnormal{det} 
				\left(
				\begin{bmatrix}
					1-\lambda_{0} 	& 0 			& 0 \\
					2\lambda_{0} 	& 1 			& 0 \\
					0 				& 2\lambda_{0} 	& 0 
				\end{bmatrix}
				\right) = 0 
			\end{equation} 
		\end{example}
		The key idea is that if pair $(E,A)$ is a regular pencil, then under an appropriate change of coordinates it is possible to simplify the problem. The idea is made more precise by the following theorem.
		\begin{theorem}[Weirstrass canonical form]
			Matrix pair $\bm{E},\bm{A}\in\mathbb{R}^{n\times n}$ is a regular pencil if, and only if, there exist invertible matrices $\bm{P}\in\mathbb{R}^{n\times n}$ and $\bm{Q}\in\mathbb{R}^{n\times n}$ such that
			\begin{equation}
			\label{eq:weierstrassform}
				(\bm{PEQ},\bm{PAQ}) = \left(
					\begin{bmatrix}
						\bm{N} & 0 \\
						0 & \bm{I}
					\end{bmatrix},
					\begin{bmatrix}
					 	\bm{I} & 0 \\
					 	0 & \bm{J} \\
					\end{bmatrix}
				\right),
			\end{equation}
			where $\bm{N}$ is an upper triangular nilpotent matrix; $\bm{J}$ is a matrix in Jordan canonical form and $\bm{I}$ is the identity matrix of suitable dimensions.
		\end{theorem}
		Just for the convenience of the reader we recall the definition of nilpotent matrix.
		\begin{definition}
			A square matrix $\bm{N}$ is said to be nilpotent if there exists an integer $\nu\in\mathbb{N}$ such that $\bm{N}^{\nu} = 0$.
		\end{definition}
		\begin{remark}
			The smallest $\nu$ that satisfies $\bm{N}^{\nu} = 0$ is usually called \emph{degree} of $\bm{N}$. Notice also that any upper triangular matrix with zeros along the main diagonal is nilpotent.
		\end{remark}
		\begin{example}
			Matrix $\bm{N}\in\mathbb{R}^{3\times 3}$ as follows is nilpotent with \emph{degree} $\nu = 3$.
			\begin{equation}
				\bm{N} = 
				\begin{bmatrix}
					0 & 1 & 3 \\
					0 & 0 & 2 \\
					0 & 0 & 0
				\end{bmatrix}, \quad
				\bm{N}^2 = 
				\begin{bmatrix}
					0 & 0 & 2 \\
					0 & 0 & 0 \\
					0 & 0 & 0
				\end{bmatrix}, \quad 
				\bm{N}^3 =
				\begin{bmatrix}
					0 & 0 & 0 \\
					0 & 0 & 0 \\
					0 & 0 & 0 
				\end{bmatrix}
			\end{equation}
		\end{example}
		The decoupling form in \cref{eq:weierstrassform} is called \emph{Weierstrass canonical form}, for further details see \cite{gantmacher1964theory}. Moreover the two matrices have the following structure; 
		\begin{equation}
			\bm{N} = 
			\begin{bmatrix}
				\bm{N}_{1} & 0 & \dots & 0 \\
				0 & \bm{N}_{2} & \dots & 0 \\
				\vdots & \vdots & \ddots & \vdots\\
				0 & 0 & \dots & \bm{N}_{n_{1}}	
			\end{bmatrix},\quad
			\bm{J} =
			\begin{bmatrix}
				\bm{J}_{1} & 0 & \dots & 0 \\
				0 & \bm{J}_{2} & \dots & 0 \\
				\vdots & \vdots & \ddots & \vdots\\
				0 & 0 & \dots & \bm{J}_{n_{2}}.
			\end{bmatrix}.
		\end{equation}
		Here $\bm{N}_{i}$ with $i=1,\dots,n_{1}$ and $\bm{J}_{i}$ with $i=1,\dots,n_{2}$ are Jordan blocks. The blocks $\bm{N}_{i}$ are associated to zero eigenvalues of matrix $\bm{E}$, therefore they appear as follows 
		\begin{equation}
			\bm{N}_{i} = 
			\begin{bmatrix}
				0 	& 1 	& 			&			&  	\\
				  	& 0 	& 1 		&			& 	\\
					&   	& \ddots 	&	\ddots	&  \\
					&   	& 			& 	\ddots	& 1 \\
					&		&			&			& 0
			\end{bmatrix}, \quad i =1,\dots,n_{1}	\\,
		\end{equation}
		and the dimension of each block $\bm{N}_{i}$ is equal to the difference between the algebraic and the geometric multiplicity of the corresponding eigenvalue. Conversely blocks $\bm{J}_{i}$ with $i=1,\dots,n_{2}$ are classical Jordan blocks with the following structure, 
		\begin{equation}	
			\bm{J}_{k} = 
				\begin{bmatrix}
					\lambda_{k} 	& 1 	& 			&			&  	\\
						& \lambda_{k} 	& 1 		&			& 	\\
						&   	& \ddots 	&	\ddots	&  \\
						&   	& 			& 	\ddots	& 1 \\
						&		&			&			& \lambda_{k}
				\end{bmatrix}, \quad k =1,\dots,n_{2}	\\.
		\end{equation}
		We now sketch a proof for the Weierstrass canonical form. The proof will be constructive and based on a sequential multiplication of non-singular matrices. Here we prove only the sufficiency part, for the necessity see \cite{gantmacher1964theory}.
		\begin{proof}
			Let us consider the regular pencil pair $(\bm{E},\bm{A})$, surely we can write $(\bm{E},\bm{A}+\lambda_{0}\bm{E}-\lambda_{0}\bm{E})$. Now let us define the matrix $\bm{T}_{0}=(\bm{A}-\lambda_{0}\bm{E})$, which is non singular by regular pencil assumption. Therefore the quantity $\bm{T}_{0}^{-1}(\bm{E},\bm{A}+\lambda_{0}\bm{E}-\lambda_{0}\bm{E}) = \left(\bm{T}_{0}^{-1}\bm{E},\bm{I}+\lambda_{0}\bm{T}_{0}^{-1}\bm{E}\right)$. Now let us call $\bm{T}_{1}$ the non singular transformation matrix such that $\bm{T}_{1}^{-1}\bm{T}_{0}^{-1}\bm{ET}_{1}$ is in Jordan canonical form. Then the pair $\bm{T}_{1}^{-1}\left(\bm{T}_{0}^{-1}\bm{E},\bm{I}+\lambda_{0}\bm{T}_{0}^{-1}\bm{E}\right)\bm{T}_{1}$ has the following structure, 
			\begin{equation}
				\label{eq:structureNJ}
				\left(
				\begin{bmatrix}
					\tilde{\bm{N}} 	& 0 \\
					0 			& \tilde{\bm{J}}
				\end{bmatrix},
				\begin{bmatrix}
					\bm{I} + \lambda_{0}\tilde{\bm{N}} 	& 0 \\
					0 							& \bm{I} + \lambda_{0}\tilde{\bm{J}}
				\end{bmatrix}
				\right).
			\end{equation}
			Here the block $\tilde{\bm{N}}$ is associated to the zero eigenvalues of the singular matrix $\bm{T}_{0}^{-1}\bm{E}$. Therefore $\tilde{\bm{N}}$ is upper triangular with zero elements on the main diagonal. Thanks to this property the matrix $(\bm{I}+\lambda_{0}\tilde{\bm{N}})$ is instead invertible. Therefore we can consider the non singular transformation matrix $\bm{T}_{2}$ defined as follows,
			\begin{equation}
				\bm{T}_{2} = 
				\begin{bmatrix}
					(\bm{I}+\lambda_{0}\tilde{\bm{N}}) & 0 \\
					0 & \tilde{\bm{J}}
				\end{bmatrix}.
			\end{equation}
			Now left multiplying \cref{eq:structureNJ} by $\bm{T}_{2}^{-1}$ we obtain the following structure,
			\begin{equation}
				\left(
					\begin{bmatrix}
						(\bm{I} + \lambda_{0}\tilde{\bm{N}})^{-1}\tilde{\bm{N}} 	& 0 \\
						0 			& \bm{I}
					\end{bmatrix},
					\begin{bmatrix}
						\bm{I} & 0 \\
						0 & \tilde{\bm{J}}^{-1}(\bm{I}+\lambda_{0}\tilde{\bm{J}}) 
					\end{bmatrix}
				\right).
			\end{equation}
			Finally it is sufficient to consider matrices $\bm{T}_{31}$ and $\bm{T}_{32}$ such that $\bm{T}_{31}^{-1}(\bm{I}+\lambda_{0}\tilde{\bm{N}})^{-1}\tilde{\bm{N}}\bm{T}_{31}$ and $\bm{T}_{32}^{-1}\tilde{\bm{J}}^{-1}(\bm{I}+\lambda_{0}\tilde{\bm{J}})\bm{T}_{32}$ are in Jordan canonical form and build matrix $\bm{T}_{3}$ as follows, 
			\begin{equation}
				\bm{T}_{3} = 
				\begin{bmatrix}
					\bm{T}_{31} & 0 \\
					0 & \bm{T}_{32}
				\end{bmatrix}.
			\end{equation}
			Thus the final change of coordinates yields,
			\begin{equation}
				\bm{T}_{3}^{-1}\left(
					\begin{bmatrix}
						(\bm{I} + \lambda_{0}\tilde{\bm{N}})^{-1}\tilde{\bm{N}} 	& 0 \\
						0 			& \bm{I}
					\end{bmatrix},
					\begin{bmatrix}
						\bm{I} & 0 \\
						0 & \tilde{\bm{J}}^{-1}(\bm{I}+\lambda_{0}\tilde{\bm{J}}) 
					\end{bmatrix}
				\right)\bm{T}_{3} = 
				\left(
					\begin{bmatrix}
						\bm{N} 	& 0 \\
						0 	& \bm{I}
					\end{bmatrix},
					\begin{bmatrix}
						\bm{I} & 0 \\
						0 & \bm{J} 
					\end{bmatrix}
				\right).
			\end{equation}
			It remains to prove that matrix $\bm{N}$ is nilpotent and to do that is sufficient to show that matrix $(\bm{I}+\lambda_{0}\tilde{\bm{N}})^{-1}\tilde{\bm{N}}$ is nilpotent as well. First notice that matrices $(\bm{I}+\lambda_{0}\tilde{\bm{N}})$ and $\tilde{\bm{N}}$ commute since, 
			\begin{equation}
				(\bm{I}+\lambda_{0}\tilde{\bm{N}})\tilde{\bm{N}} = \tilde{\bm{N}} + \lambda_{0}\tilde{\bm{N}}^{2} = \tilde{\bm{N}}(\bm{I}+\lambda_{0}\tilde{\bm{N}}).
			\end{equation} 
			Then also $(\bm{I}+\lambda_{0}\tilde{\bm{N}})^{-1}$ and $\tilde{\bm{N}}$ commute since, 
			\begin{equation}
				(\bm{I}+\lambda_{0}\tilde{\bm{N}})^{-1}\tilde{\bm{N}} = (\bm{I}+\lambda_{0}\tilde{\bm{N}})^{-1}\tilde{\bm{N}}(\bm{I}+\lambda_{0}\tilde{\bm{N}})(\bm{I}+\lambda_{0}\tilde{\bm{N}})^{-1} = \tilde{\bm{N}}(\bm{I}+\lambda_{0}\tilde{\bm{N}})^{-1}
			\end{equation}
			where we used the fact that $\tilde{\bm{N}}(\bm{I}+\lambda_{0}\tilde{\bm{N}})=(\bm{I}+\lambda_{0}\tilde{\bm{N}})\tilde{\bm{N}}$. Now since we know that commutative property for these matrices holds, called $\nu$ the index such that $\tilde{\bm{N}}^{\nu}=0$ the following is true
			\begin{equation}
				\left[(\bm{I}+\lambda_{0}\tilde{\bm{N}})\tilde{\bm{N}}\right]^{\nu} = (\bm{I}+\lambda_{0}\tilde{\bm{N}})^{\nu}\tilde{\bm{N}}^{\nu}=0,
			\end{equation} and this complete the proof. Just to summarize, we notice that the overall transformation matrices $(\bm{P},\bm{Q})$ are given by $\bm{P} = (\bm{T}_{0}\bm{T}_{1}\bm{T}_{2}\bm{T}_{3})^{-1}$ and $\bm{Q} = \bm{T}_{1}\bm{T}_{3}$.
		\end{proof}
		The Weirstrass canonical form is quite useful since we can re-write system \cref{eq:daelinear} using the transformation matrices $(\bm{P},\bm{Q})$ just introduced, thus
		\begin{equation}
				\bm{PEQQ^{-1}}\dot{\bm{x}}(t) = \bm{PAQQ}^{-1}\bm{x}(t)+\bm{Pf}(t).
		\end{equation}
		Now considering the following change of coordinates,
		\begin{subequations}
			\begin{align}
				\label{eq:changeofcoordinates}
				\bm{z}(t) &= \bm{Q}^{-1}\bm{x}(t), \\
				\bm{g}(t) &= \bm{Pf}(t),
			\end{align}
		\end{subequations}
		we obtain the following equivalent DAE s
		\begin{equation}
			\begin{bmatrix}
				\bm{N} & 0 \\
				0 & \bm{I}
			\end{bmatrix}
			\begin{bmatrix}
				\dot{\bm{z}}_{1}(t) \\
				\dot{\bm{z}}_{2}(t)
			\end{bmatrix} 
			= 
			\begin{bmatrix}
				\bm{I} & 0 \\
				0 & \bm{J}
			\end{bmatrix}
			\begin{bmatrix}
				\bm{z}_{1}(t) \\
				\bm{z}_{2}(t)
			\end{bmatrix}
			+ 
			\begin{bmatrix}
				\bm{g}_{1}(t) \\
				\bm{g}_{2}(t)
			\end{bmatrix}.
		\end{equation}
		Here we partitioned the $\bm{z}(t)$ and $\bm{g}(t)$ vectors according respectively to the dimensions of  $\bm{J}$ and $\bm{N}$. Obviously $\bm{z}(t) = (\bm{z}_{1}(t);\bm{z}_{2}(t))$ and $\bm{g}(t) = (\bm{g}_{1}(t);\bm{g}_{2}(t))$. It is also interesting, notice that the two sets of equations corresponding to $\bm{z}_{1}(t)$ and $\bm{z}_{2}(t)$ are decoupled
		\begin{subequations}
			\begin{align}
				\label{eq:stillDAE}
				\bm{N}\dot{\bm{z}}_{1} 	&= \bm{z}_{1} + \bm{g}_{1}(t), \\
				\dot{\bm{z}}_{2} 	&= \bm{J}\bm{z}_{2} + \bm{g}_{2}(t).
			\end{align}
		\end{subequations}
		It is straightforward to see that the equation associated to $\bm{z}_{2}(t)$ is just an ODE then the unique solution  from the initial condition $\bm{z}_{2}(0)$ is given by
		\begin{equation}
			\bm{z}_{2}(t) = e^{\bm{J}t}\bm{z}_{2}(0)+\int_{0}^{t}e^{\bm{J}(t-\tau)}\bm{g}_{2}(\tau)\diff\tau,\quad \forall t\geq 0,
		\end{equation}
		(for further details see for example \cite{hespanha2009linear}). \cref{eq:stillDAE} is instead again a DAE, but since matrix $N$ is nilpotent we can manipulate it. It is convenient to re-write \cref{eq:stillDAE} as
		\begin{equation}
			\label{eq:subsitself}
			\bm{z}_{1}(t) = \bm{N}\dot{\bm{z}}_{1}(t) - \bm{g}_{1}(t).
		\end{equation}
		Now substituting \cref{eq:subsitself} inside itself and 
		differentiating yields,
		\begin{subequations}
			\begin{align*}
				\bm{z}_{1}(t) &= \bm{N}\frac{\diff}{\diff t}\left[\bm{N}\dot{\bm{z}}_{1}(t)-\bm{g}_{1}(t)\right]-\bm{g}_{1}(t) = \bm{N}^{2}\frac{\diff^{2}\bm{z}_{1}(t)}{\diff t^{2}}-\bm{N}\frac{\diff \bm{g}_{1}(t)}{\diff t}-\bm{g}_{1}(t) \\
				  &= \bm{N}^{2}\frac{\diff^{2}}{\diff t^{2}}\left[\bm{N}\dot{\bm{z}}_{1}(t)-\bm{g}_{1}(t)\right]-\bm{N}\frac{\diff \bm{g}_{1}(t)}{\diff t}-\bm{g}_{1}(t) = \bm{N}^3\frac{\diff^{3}\bm{z}_{1}(t)}{\diff t^{3}}-\bm{N}^2\frac{\diff^{2} \bm{g}_{1}(t)}{\diff t^{2}}-\dots \\
				  &\vdots \\
				  & = \bm{N}^{\nu}\frac{\diff^{\nu}\bm{z}_{1}(t)}{\diff t^{\nu}} - \bm{N}^{\nu-1}\frac{\diff^{\nu-1} \bm{g}_{1}(t)}{\diff t^{\nu-1}}-\dots-\bm{N}\frac{\diff \bm{g}_{1}(t)}{\diff t} - \bm{g}_{1}(t),
			\end{align*}
		\end{subequations}
		establishing differential properties of the solutions. Notice that the procedure stops when we reach the degree $\nu$ of matrix $\bm{N}$, indeed $\bm{N}^{\nu} = 0$ and we may find a unique solution expressed as a finite summation of increasing derivatives of $\bm{g}(t)$ as follows:
		\begin{equation}
			\bm{z}_{1}(t) = -\sum_{k=0}^{\nu-1}\bm{N}^{k}\frac{\diff^{k} \bm{g}_{1}(t)}{\diff t^{k}}.
		\end{equation}
		\begin{remark}
			Sometimes index $\nu$ is also called \emph{Kronecker index}. Notice also that $\nu$ is the number of times that we differentiated function $\bm{g}_{1}(t)$ in order to obtain the solution $\bm{z}_{1}(t)$.
		\end{remark}
		Finally we are ready to express the unique solution to the original problem in \cref{eq:daelinear}. It is sufficient to invert the change of coordinates in \cref{eq:changeofcoordinates} and we obtain
		\begin{equation}
			\label{eq:solutionDAE}
			\bm{x}(t) = \bm{Q}\bm{z}(t) = \bm{Q}
			\begin{bmatrix}
				-\sum_{k=0}^{\nu-1}\bm{N}^{k}\frac{\diff^{k} \bm{g}_{1}(t)}{\diff t^{k}} \\
				e^{\bm{J}t}\bm{z}_{2}(0)+\int_{0}^{t}e^{\bm{J}(t-\tau)}\bm{g}_{2}(\tau)\diff\tau
			\end{bmatrix}
		\end{equation}
		\begin{remark}
			Notice that the initial condition $\bm{z}_{1}(0)$ cannot be chosen freely, but has to satisfy the following,
			\begin{equation}
				\bm{z}_{1}(0) = -\sum_{k=0}^{\nu-1}\bm{N}^{k}\frac{\diff^{k}\bm{g}_{1}(0)}{\diff t^{k}}.
			\end{equation}
			Therefore also the initial condition $\bm{x}_{0}=\bm{x}(t=0)$ for \cref{eq:daelinear} cannot be chosen completely freely. This intuitively expresses the fact that also the initial condition has to be consistent with constrains expressed by $\bm{E}$.
		\end{remark}
		Summarizing what we discussed we have the following,
		\begin{theorem}
			Problem in \cref{eq:daelinear} with initial condition $\bm{x}_{0} = \bm{x}(t=0)$ has a solution if the following conditions holds:
			\begin{itemize}
				\item function $\bm{f}(t)$ is piece-wise continuous,
				\item the pair $(\bm{E,A})$ is a regular pencil,
				\item the initial condition $\bm{x}_{0}$ satisfies 
				$\bm{x}_{0} = \bm{Q}\begin{bmatrix} -\sum_{k=0}^{\nu-1}\bm{N}^{k}\frac{\diff^{k} \bm{g}_{1}(0)}{\diff t^{k}} \\ \bm{z}_{2}(0) \end{bmatrix}$.
			\end{itemize}
			Moreover the corresponding solutions is given by \cref{eq:solutionDAE}.
		\end{theorem}
	\section{Review of Gaussian Elimination}
		The Weirstrass canonical form is very useful from the theoretical point of view, however is computationally demanding, because it requires to go through several Jordan canonical forms. A numerical more efficient algorithm to reduce a DAE to an ODE is presented in the next section. The idea is to use the celebrated Gauss's elimination method combined with differentiation in order to perform the desired reduction. Before doing so, we want recall some elementary concepts and definitions useful for the purpose. 
		\begin{definition}
			Given a matrix $\bm{M}\in\mathbb{R}^{n\times m}$ we call elementary operations the followings,
			\begin{itemize}
				\item permutation of two rows (resp. columns),
				\item linear combinations of rows (resp. columns). 
			\end{itemize} 
		\end{definition}
		These operations can be also regarded as matrix multiplications.
		\begin{definition}
			A non-singular matrix $\bm{P}(i,j)\in\left\lbrace0,1\right\rbrace^{n\times n}$ (resp. $\bm{P}(i,j)\in\left\lbrace0,1\right\rbrace^{m\times m}$) is said to be a rows (resp. columns) permutation matrix if it is obtained by the identity matrix $\bm{I}\in\mathbb{R}^{n\times n}$ (resp. $\bm{I}\in\mathbb{R}^{m\times m}$) exchanging the column $i$ with column $j$.
		\end{definition}
		Permutation matrices have the following remarkable properties,
		\begin{itemize}
			\item $\bm{P}(i,j)^{-1} = \bm{P}(i,j)^{\intercal}$, therefore $\bm{P}(i,j)$ is an orthogonal matrix.
			\item $\bm{P}(i,j)^{2}=\bm{I}$, therefore $\bm{P}(i,j)$ is an involution matrix.
			\item $\bm{P}(i,j)\bm{M}$ exchanges row $i$ with row $j$. 
			\item $\bm{MP}(i,j)$ exchanges column $i$ with column $j$.
		\end{itemize}
		\begin{example}
			Consider the following permutation matrix $\bm{P}(1,2)\in\left\lbrace 1,0\right\rbrace^{3\times 3}$ and a matrix $\bm{M}\in\mathbb{R}^{3\times 3}$ then $\bm{P}(1,2)\bm{M}$ exchanges rows $(1,2)$ of matrix $\bm{M}$.
			\begin{equation}
			 	\bm{P}(1,2)\bm{M} = 
			 	\begin{bmatrix}
			 		0 & 1 & 0 \\
			 		1 & 0 & 0 \\
			 		0 & 0 & 1 
			 	\end{bmatrix}
			 	\begin{bmatrix}
			 		1 & 2 & 3 \\
			 		4 & 5 & 6 \\
			 		7 & 8 & 9
			 	\end{bmatrix}
			 	= 
			 	\begin{bmatrix}
			 		4 & 5 & 6 \\
			 		1 & 2 & 3 \\
			 		7 & 8 & 9
			 	\end{bmatrix}
			\end{equation}
			In the same way $\bm{MP}(1,2)$ exchanges columns $(1,2)$ of matrix $\bm{M}$,
			\begin{equation}
			 	\bm{MP}(1,2) = 
			 	\begin{bmatrix}
			 		1 & 2 & 3 \\
			 		4 & 5 & 6 \\
			 		7 & 8 & 9
			 	\end{bmatrix}
			 	\begin{bmatrix}
			 		0 & 1 & 0 \\
			 		1 & 0 & 0 \\
			 		0 & 0 & 1 
			 	\end{bmatrix}
			 	= 
			 	\begin{bmatrix}
			 		2 & 1 & 3 \\
			 		5 & 4 & 6 \\
			 		8 & 7 & 9
			 	\end{bmatrix}
			\end{equation}
			
		\end{example}
		In analogous way it is possible to define a matrix that performs linear combinations of rows (resp. columns). These matrices are a special class of Frobenius matrices and usually are called elementary matrices. 
		\begin{definition}
			A matrix $\bm{F}(i,j,\alpha)\in\mathbb{R}^{n\times n}$ (resp. $\bm{F}(i,j,\alpha)\in\mathbb{R}^{m\times m}$) is said to be a rows (resp. columns) elementary matrix if it has the following structure,
			\begin{equation}
				\bm{F}(i,j,\alpha) =
				\begin{bmatrix}
					1  		& 	\dots 	& 0 		& \dots 	& 0 		& \dots 	& 0 \\
					\vdots 	&	\ddots	& \vdots 	& 			& \vdots	& 			& \vdots \\
					0		& 	\dots	& 1			& 			& 0			& 			& 0 \\
					\vdots	& 			& \vdots 	& \ddots 	& \vdots 	& 			& \vdots \\ 	
					0		& \dots		& \alpha 	& \dots		& 1			& \dots		& 0 \\
					\vdots	& 			& \vdots	& 			& \vdots	& \ddots	& \vdots \\
					0 		& \dots		& 0			& \dots		& 0			& \dots 	& 1
				\end{bmatrix}
			\end{equation}
			with the $(i,j)$ element equal to $\alpha$.	
		\end{definition}
		\begin{remark}
			An elementary matrix $\bm{F}(i,j,\alpha)\in\mathbb{R}^{n\times n}$ (resp. $\bm{F}(i,j,\alpha)\in\mathbb{R}^{m\times m}$) is equal to the identity matrix of dimension $n$ (resp. $m$) for all the entries except the $(i,j)$ position which is equal to $\alpha$. 
		\end{remark}
		Combination matrices have the following remarkable properties:
		\begin{itemize}
			\item $\bm{F}(i,j,\alpha)\bm{M}$ add the row $j$ multiplied by $\alpha$ to row $i$,
			\item $\bm{MF}(i,j,\alpha)$ add the column $i$ multiplied by $\alpha$ to column $j$.
		\end{itemize}
		\begin{theorem}
			\label{th:gausssimply}
			Given a matrix $\bm{M}\in\mathbb{R}^{n\times m}$ with $n\leq m$ and $\textnormal{rank}~\bm{M} = r\leq n$, there exist a pair of non singular matrices, $\bm{L}\in\mathbb{R}^{n\times n}$ and $\bm{U}\in\mathbb{R}^{m\times m}$ such that,
			\begin{equation}
				\bm{LMU} = 
				\begin{bmatrix}
					\bm{\Lambda} & 0 \\
					0 		& 0
				\end{bmatrix},
			\end{equation}
			with $\bm{\Lambda}\in\mathbb{R}^{r\times r}$ a non singular diagonal  matrix. Blocks of zeros have dimensions compatible with the partitioning.
		\end{theorem}
		\begin{proof}
			The proof is quite algorithmic and based on sequentially Gauss's eliminations. We define $\bm{M}^{(k)}$ the matrix $\bm{M}$ at the $k$ iteration. Parenthesis are used to stress out that $\bm{M}^{(k)}$ is not the matrix $\bm{M}$ to the power $k$. So let us initialize the algorithm with $\bm{M}^{(1)} = \bm{M}$ and $k=1$. Now the idea is to look for the firs non zero element inside the minor $\bm{M}(k\dots n, k\dots m)$; obviously at the first iteration we are exploring all the entries. Suppose to move first along rows and next along columns; then unless the matrix is completely full of zeros we will find a pair of indices such that $m^{(k)}_{pq}\neq 0$. Then the idea is to move this non zero element in the $(k,k)$ position. This can be done with a rows and columns permutations as follows, 
			\begin{equation}
				\overline{\bm{M}}^{(k)} = \bm{P}(k,p)\bm{M}^{(k)}\bm{P}(q,k).
			\end{equation} 
			Notice that the $\overline{m}^{(k)}_{k,k} = m^{k}_{p,q}$, roughly speaking the non zero element in position $(p,q)$ of matrix $\bm{M}^{(k)}$ is moved to the $(k,k)$ position of matrix $\overline{\bm{M}}^{(k)}$. Now we can use this non zero element to set to zero all the element along the $k$ column and the $k$ row. This can be done using a composition of linear combination matrices. Specifically the matrix $\overline{\bm{L}}^{(k)}\in\mathbb{R}^{n\times n}$ defined as follows is designed to set to zero all the elements in the indices range $(k\dots n, k\dots k)$,
			\begin{equation}
				\begin{split}
					\overline{\bm{L}}^{(k)} &= \bm{F}\left(k+1,k,-\frac{\overline{m}_{k+1,k}}{\overline{m}_{k,k}}\right)\dots \bm{F}\left(n-1,k,-\frac{\overline{m}_{n-1,k}}{\overline{m}_{k,k}}\right)\bm{F}\left(n,k,-\frac{\overline{m}_{n,k}}{\overline{m}_{k,k}}\right) \\
					&= \prod_{h=k+1}^{n} \bm{F}\left(h,k,-\frac{\overline{m}_{h,k}}{\overline{m}_{k,k}}\right)
				\end{split}.
			\end{equation}
			The same operation can be performed along columns, in this case matrix $\overline{\bm{U}}^{(k)}$ performs linear combinations to set to zero all the element in the indices range $(k\dots k, k\dots m)$, 
			\begin{equation}
				\begin{split}
					\overline{\bm{U}}^{(k)} &= \bm{F}\left(k,k+1,-\frac{\overline{m}_{k,k+1}}{\overline{m}_{k,k}}\right)\dots \bm{F}\left(k,n-1,-\frac{\overline{m}_{k,n-1}}{\overline{m}_{k,k}}\right)\bm{F}\left(k,n,-\frac{\overline{m}_{k,n}}{\overline{m}_{k,k}}\right) \\
					&= \prod_{h=k+1}^{m} \bm{F}\left(k,h,-\frac{\overline{m}_{k,h}}{\overline{m}_{k,k}}\right).
				\end{split}
			\end{equation}
			Now just for notational convenience  let us define the following quantities, 
			\begin{subequations}
				\begin{align}
					\bm{L}^{(k)} &= \overline{\bm{L}}^{(k)}\bm{P}(k,p), \\
					\bm{U}^{(k)} &= \bm{P}(q,k)\overline{\bm{U}}^{(k)}.
				\end{align}
			\end{subequations}
			Then using simultaneously the two matrices $\bm{L}^{(k)}\in\mathbb{R}^{n\times n}$ and $\bm{U}^{(k)}\in\mathbb{R}^{m\times m}$ we can set to zero all the elements along the $k$ row and the $k$ column, obviously except the $(k,k)$ position. Then the matrix $M$ at next iteration is given by,
			\begin{equation}
				\bm{M}^{(k+1)} = \bm{L}^{(k)}\bm{M}^{(k)}\bm{U}^{(k)}.
			\end{equation}
			Is now easy to see that iterating the process; after exactly $r$ iterations, matrix $\bm{M}^{(r+1)}\in\mathbb{R}^{n\times m}$ has exactly the following structure,
			\begin{equation}
				\bm{M}^{(r+1)} = 
				\begin{bmatrix}
					\bm{\Lambda} & 0 \\
					0 		& 0 
				\end{bmatrix}.
			\end{equation}
			Moreover the matrices $\bm{L}\in\mathbb{R}^{n\times n}$ and $\bm{U}\in\mathbb{R}^{m\times m}$ that realize that structure are given by, 
			\begin{subequations}
				\begin{align}
					\bm{L} &= \bm{L}^{(r)}\bm{L}^{(r-1)}\dots\bm{L}^{(1)}, \\
					\bm{U} &= \bm{U}^{(1)}\bm{U}^{(2)}\dots \bm{U}^{(r)}.
				\end{align}
			\end{subequations}
			Notice also that $\bm{L}$ and $\bm{U}$ are non singular since are obtained through product of non singular matrices. 
		\end{proof}
		To make things clear we propose the following example of the algorithm.
		\begin{example}
			Let us consider the matrix $\bm{M}\in\mathbb{R}^{3\times 4}$ with $\textnormal{rank}~\bm{M} = r = 2$ defined as follows,
			\begin{equation}
				\bm{M} = 
				\begin{bmatrix}
					0 & 0 & 1 & 0 \\
					0 & 2 & 0 & 0 \\
					0 & 4 & 2 & 0
 				\end{bmatrix}.
			\end{equation}
			First of all we set $\bm{M}^{(1)} = \bm{M}$ and $k=1$ and we look for the firs non zero element starting from position $(k,k) = (1,1)$ and moving along the first column and then along the second and so forth. At the first iteration the first non zero element is in the position $(p,q) = (2,2)$, so the following permutation matrices $\bm{P}(1,2)\in\left\lbrace0,1 \right\rbrace^{3\times 3}$ and $\bm{P}(2,1)\in\left\lbrace 0,1 \right\rbrace^{4\times 4}$ provide a way to move it in the $(1,1)$ position. 
			\begin{equation}
					\overline{\bm{M}}^{(1)} = \bm{P}(1,2)\bm{M}^{(1)}\bm{P}(2,1) = 
					\begin{bmatrix}
						0 & 1 & 0 \\
						1 & 0 & 0 \\
						0 & 0 & 1 \\
					\end{bmatrix}
					\bm{M}^{(1)}
					\begin{bmatrix}
						0 & 1 & 0 & 0 \\
						1 & 0 & 0 & 0 \\
						0 & 0 & 1 & 0 \\
						0 & 0 & 0 & 1
					\end{bmatrix} \\
					=  
					\begin{bmatrix}
						2 & 0 & 0 & 0 \\
						0 & 0 & 1 & 0 \\
						4 & 0 & 2 & 0 \\
					\end{bmatrix}
			\end{equation}
			Now we need to set to zero all the non zero elements in column and row $1$, and the non only non zero element is $\overline{m}_{3,1}^{(1)} = 4$. Therefore the row combination matrix $\bm{C}(3,1,-4/2)\in\mathbb{R}^{3\times 3}$ reaches the goal,
			\begin{equation}
				\begin{split}
					\bm{M}^{(2)} 	&= \bm{C}(3,1,-4/2)\overline{\bm{M}}^{(1)}  
					\begin{bmatrix}
						1 	& 	0 	& 0 \\
						0 	& 	1 	& 0 \\
						-2 	& 	0 	& 1
					\end{bmatrix} 
					\begin{bmatrix}
						2 & 0 & 0 & 0 \\
						0 & 0 & 1 & 0 \\
						4 & 0 & 2 & 0 \\
					\end{bmatrix}
					=
					\begin{bmatrix}
						2 & 0 & 0 & 0 \\
						0 & 0 & 1 & 0 \\
						0 & 0 & 2 & 0
					\end{bmatrix}
				\end{split}
			\end{equation}
			Now  $k=2$ and we perform again a search for the first non zero element in $\bm{M}^{(2)}$. The answer is $(p,q)=(2,3)$ so we can use again permutation matrices, $\bm{P}(2,2) = I\in\left\lbrace0,1 \right\rbrace^{3\times 3}$ and $\bm{P}(3,1)\in\left\lbrace0,1 \right\rbrace^{4\times 4}$.
			\begin{equation}
				\overline{\bm{M}}^{(2)} = \bm{M}^{(2)}\bm{P}(3,1) = 							\begin{bmatrix}
					2 & 0 & 0 & 0 \\
					0 & 0 & 1 & 0 \\
					0 & 0 & 2 & 0
				\end{bmatrix}
				\begin{bmatrix}
     				1 & 0 & 0 & 0 \\
     				0 & 0 & 1 & 0 \\
     				0 & 1 & 0 & 0 \\
     				0 & 0 & 0 & 1 
				\end{bmatrix}
				= 
				\begin{bmatrix}
					2 & 0 & 0 & 0 \\
					0 & 1 & 0 & 0 \\
					0 & 2 & 0 & 0 
				\end{bmatrix}
			\end{equation}
			Finally we can use the rows combination matrix $\bm{C}(3,2,-2/1)\in\mathbb{R}^{3\times 3}$ to complete the procedure,
			\begin{equation}
				\bm{M}^{(3)} = \bm{C}(3,2,-2/1)\overline{\bm{M}}^{(2)} = 
				\begin{bmatrix}
					1 & 0  & 0 \\
					0 & 1  & 0 \\
					0 & -2 & 1
				\end{bmatrix}
				\begin{bmatrix}
					2 & 0 & 0 & 0 \\
					0 & 1 & 0 & 0 \\
					0 & 2 & 0 & 0 
				\end{bmatrix}
				= 
				\begin{bmatrix}
					2 & 0 & 0 & 0 \\
					0 & 1 & 0 & 0 \\
					0 & 0 & 0 & 0 
				\end{bmatrix}
			\end{equation}
			Here obviously the diagonal block $\bm{\Lambda}\in\mathbb{R}^{2\times 2} = \textnormal{diag}(2,1)$. Notice also that is now clear that the rank of matrix $M$ is equal $r = 2$.
		\end{example}
	\section{Gauss like method to compute DAEs' index}
		We are finally ready to discuss why \cref{th:gausssimply} is useful in computing the index of a DAE, and how it can be used to transform a generic linear DAE into an ODE. So let us consider a generic linear DAE in the form shown in \cref{eq:daelinear}, and we notice that can be also written as, 
		\begin{equation}
			\label{eq:computationalDAE}
			\begin{bmatrix}
				\bm{E} & -\bm{A} 
			\end{bmatrix}
			\begin{bmatrix}
				\dot{\bm{x}} \\
				\bm{x}
			\end{bmatrix}
			= \bm{f}.
		\end{equation}
		Just for notational brevity we dropped the time dependency. Here matrix $\begin{bmatrix}\bm{E} & -\bm{A}\end{bmatrix}\in\mathbb{R}^{n\times 2n}$. Let us call
		\begin{equation}
			\textnormal{rank}~\bm{E} =: r \leq n.
		\end{equation}
		Then let us consider a pair of matrices $\bm{L}^{(0)}\in\mathbb{R}^{n\times n}$ and $\bm{U}^{(0)}\in\mathbb{R}^{n\times n}$ such that,
		\begin{equation*}
			\bm{U}^{(0)}\bm{E}\bm{L}^{(0)} = \bm{E}^{(0)} = 
			\begin{bmatrix}
				\bm{\Lambda}^{(0)} 	& 0 \\
				0 				& 0 
			\end{bmatrix}.
		\end{equation*}
		As done in the previous section we used the superscript $^{(0)}$ to indicate the iteration number; it will be clear soon that this number is also the Kronecker index of the DAE. Notice that at this first iteration $\bm{\Lambda}^{(0)}\in\mathbb{R}^{r\times r}$. If $r=n$ then we done since the DAE has index $\nu = 0$ and matrix $\bm{E}$ was full rank and invertible. If $r<n$ it is convenient introduce the following change of coordinates, 
		\begin{equation}
			\begin{bmatrix}
				\dot{\bm{x}}^{(0)} \\
				\bm{x}^{(0)} 
			\end{bmatrix}
			=
			\begin{bmatrix}
				\bm{U}^{(0)} & 0 \\
				0 & \bm{I} 
			\end{bmatrix}^{-1}
			\begin{bmatrix}
				\dot{\bm{x}} \\
				\bm{x}
			\end{bmatrix},
		\end{equation}
		and also the matrix $\bm{A}^{(0)}$ and vector $\bm{f}^{(0)}$ as follows,
		\begin{subequations}
			\begin{align*}
				\bm{A}^{(0)} 	&:= -\bm{L}^{(0)}\bm{A},\\
				\bm{f}^{(0)} 	&:= \bm{L}^{(0)}\bm{f}.
			\end{align*}
		\end{subequations}
		Let us consider to multiply on the left-hand side \cref{eq:computationalDAE} by $\bm{L}^{(0)}$. Then we obtain the following equivalent form,
		\begin{equation*}
			\begin{split}
				\bm{L}^{(0)}
				\begin{bmatrix}
					\bm{E} & -\bm{A} 
				\end{bmatrix}
				\begin{bmatrix}
					\bm{U}^{(0)} & 0 \\
					0 & \bm{I} 
				\end{bmatrix}
				\begin{bmatrix}
					\bm{U}^{(0)} & 0 \\
					0 & \bm{I} 
				\end{bmatrix}^{-1}
				\begin{bmatrix}
					\dot{\bm{x}} \\
					\bm{x}
				\end{bmatrix}
				&= 
				\begin{bmatrix}
					\bm{L}^{(0)}\bm{EU}^{(0)} & \bm{L}^{(0)}\bm{A}
				\end{bmatrix}
				\begin{bmatrix}
					\dot{\bm{x}}^{(0)}\\
					\bm{x}^{(0)}
				\end{bmatrix} \\
				&= 
				\begin{bmatrix}
					\bm{E}^{(0)} & \bm{A}^{(0)}
				\end{bmatrix}
				\begin{bmatrix}
					\dot{\bm{x}}^{(0)} \\
					\bm{x}^{(0)} \\
				\end{bmatrix}
				=
				\bm{f}^{(0)}(t)
			\end{split}
		\end{equation*}
		Now notice that matrix $\begin{bmatrix}\bm{E}^{(0)} & \bm{A}^{(0)}\end{bmatrix}\in\mathbb{R}^{n\times 2n}$ has the following special structure, 
		\begin{equation}
			\label{eq:specialDAE}
			\begin{bmatrix}
				\bm{\Lambda}^{(0)}	& 0 & \bm{A}_{11}^{(0)} 	& \bm{A}_{12}^{(0)} \\
				0 					& 0 & \bm{A}_{21}^{(0)}  	& \bm{A}_{22}^{(0)}
			\end{bmatrix}
			\begin{bmatrix}
				\dot{\bm{x}}_{1}^{(0)}\\
				\dot{\bm{x}}_{2}^{(0)}\\
				\bm{x}_{1}^{(0)}\\
				\bm{x}_{2}^{(0)} \\
			\end{bmatrix}
			= 
			\begin{bmatrix}
				\bm{f}_{1}^{(0)}\\
				\bm{f}_{2}^{(0)}
			\end{bmatrix}.
		\end{equation}
		Obviously \cref{eq:specialDAE} can be also written as a couple of matrix equations as follows,
		\begin{subequations}
			\begin{align}
				\label{eq:diffpartDAE}
				\bm{\Lambda}^{(0)}\dot{\bm{x}}_{1}^{(0)} + \bm{A}_{11}^{(0)}\bm{x}_{1}^{(0)} + \bm{A}_{12}^{(0)}\bm{x}_2^{(0)} &= \bm{f}_{1}^{(0)}, \\
				\label{eq:algebraicpartDAE}
				\bm{A}_{21}^{(0)}\bm{x}_{1}^{(0)} + \bm{A}_{22}^{(0)}\bm{x}_{2}^{(0)} &= \bm{f}_{2}^{(0)},
			\end{align}
		\end{subequations}
		where $\dot{\bm{x}}_{1}^{(0)},\bm{x}_{1}^{(0)}\in\mathbb{R}^{r}$, $\dot{\bm{x}}_{2}^{(0)},\bm{x}_{2}^{(0)}\in\mathbb{R}^{n-r}$ and $\bm{x}^{(0)} = [\bm{x}_{1}^{(0)}; \bm{x}_{2}^{(0)}]$. Notice that \cref{eq:diffpartDAE} is a ODE since $\bm{\Lambda}^{(0)}$ is invertible, on the contrary \cref{eq:algebraicpartDAE} is a pure algebraic equation. Therefore in these coordinates the differential and the algebraic components of the DAE are decoupled. We now differentiate with respect to time \cref{eq:algebraicpartDAE} and we obtain the following, 
		\begin{equation}
			\label{eq:constrafterdiff}
			\bm{A}_{21}^{(0)}\dot{\bm{x}}_{1}^{(0)} + \bm{A}_{22}^{(0)}\dot{\bm{x}}_{2}^{(0)} = \dot{\bm{f}}_{2}^{(0)}.
		\end{equation} 
		Writing now \cref{eq:constrafterdiff} with \cref{eq:diffpartDAE} in matrix form we obtain the following, 
		\begin{equation}
			\label{eq:DAEafterfirstdiff}
			\begin{bmatrix}
				\bm{\Lambda}^{(0)}	& 0 				& \bm{A}_{11}^{(0)} 	& \bm{A}_{12}^{(0)} \\
				\bm{A}_{21}^{(0)}	& \bm{A}_{22}^{(0)}	& 0 					& 0    
			\end{bmatrix}
			\begin{bmatrix}
				\dot{\bm{x}}_{1}^{(0)}\\
				\dot{\bm{x}}_{2}^{(0)} \\
				\bm{x}_{1}^{(0)}\\
				\bm{x}_{2}^{(0)}\\
			\end{bmatrix}
			= 
			\begin{bmatrix}
				\bm{v}_{1}^{(0)}\\
				\dot{\bm{f}}_{2}^{(0)}
			\end{bmatrix}.
		\end{equation}
		As we can observe the differentiation w.r.t time of the algebraic constraints in the matrix representation produces just a shift to the left of the two blocks $\bm{A}_{21}^{(0)}$, $\bm{A}_{22}^{(0)}$. Notice also this shift increases the rank of the left block as it is clearly stated below,
		\begin{equation}
				\textnormal{rank}~
				\begin{bmatrix}
					\bm{\Lambda}^{(0)} 	& 0 \\
					0 					& 0 
				\end{bmatrix} <
				\textnormal{rank}~
				\begin{bmatrix}
					\bm{\Lambda}^{(0)}	& 0 \\
					\bm{A_{21}}^{(0)}	& \bm{A_{22}}^{(0)}
				\end{bmatrix}
		\end{equation}
		Now we can apply again \cref{th:gausssimply} and find a pair of matrices $\bm{L}^{(1)}\in\mathbb{R}^{n\times n}$ and $\bm{U}^{(1)}\in\mathbb{R}^{n\times n}$ such that,
		\begin{equation*}
			\bm{U}^{(1)}
			\begin{bmatrix}
				\bm{\Lambda}^{(0)}	& 0 \\
				\bm{A}_{21}^{(0)}	& \bm{A}_{22}^{(0)}
			\end{bmatrix}
			\bm{L}^{(1)} 
			= \bm{E}^{(1)} = 
			\begin{bmatrix}
				\bm{\Lambda}^{(1)} 	& 0 \\
				0 					& 0 
			\end{bmatrix},
		\end{equation*}
		and perform another change of coordinates as follows,
		\begin{subequations}
			\label{eq:secondchangeofcoordinates}
			\begin{align*}
				\begin{bmatrix}
					\dot{\bm{x}}^{(1)} \\
					\bm{x}^{(1)} 
				\end{bmatrix}
				&=
				\begin{bmatrix}
					\bm{U}^{(1)} & 0 \\
					0 & \bm{I} 
				\end{bmatrix}^{-1}
				\begin{bmatrix}
					\dot{\bm{x}}^{(0)} \\
					\bm{x}^{(0)}
				\end{bmatrix}, \\
				\bm{A}^{(1)} 		&= \bm{L}^{(1)}\bm{A}^{(0)}, \\
				\bm{f}^{(1)} 		&= \bm{L}^{(0)}
				\begin{bmatrix}
					\bm{f}_{1}^{(0)} \\
					\dot{\bm{f}}_{2}^{(0)}
				\end{bmatrix}.
			\end{align*}
		\end{subequations}
		Please notice that the definition of $\bm{f}^{(1)}$ contains part of the field $\bm{f}^{(0)}$ but also some derivatives of it. Now we can left multiply \cref{eq:DAEafterfirstdiff} by $\bm{L}^{(1)}$  and use the change of coordinates NO QUESTO NON VA BENE SEGNO - defined in \cref{eq:secondchangeofcoordinates}. So finally we reach the following form, 
		\begin{equation}
			\begin{bmatrix}
				\bm{E}^{(1)} & \bm{A}^{(1)} 
			\end{bmatrix}
			\begin{bmatrix}
				\dot{\bm{x}}^{(1)} \\
				\bm{x}^{(1)}
			\end{bmatrix} = \bm{f}^{(1)}.
		\end{equation}
		Here matrix $\begin{bmatrix}\bm{E}^{(1)} & \bm{A}^{(1)}\end{bmatrix}$ as seen before has the following structure, 
		\begin{equation}
			\begin{bmatrix}
				\bm{E}^{(1)} & \bm{A}^{(1)}
			\end{bmatrix}
			=
			\begin{bmatrix}
				\bm{\Lambda}^{(1)} 	& 0 & \bm{A}_{11}^{(1)} & \bm{A}_{12}^{(1)}  \\
				0 					& 0 & \bm{A}_{21}^{(1)} & \bm{A}_{22}^{(1)}
			\end{bmatrix}.
		\end{equation}
		It is important to point out that the dimensions of matrix $\bm{\Lambda}^{(1)}$ are greater than the ones of matrix $\bm{\Lambda}^{(0)}$, formally
		\begin{equation}
			\textnormal{dim}~\bm{\Lambda}^{(1)} > \textnormal{dim}~\bm{\Lambda}^{(0)}.
		\end{equation}
		It is therefore clear that at each iteration (derivation) of the constraints the dimensions of $\bm{\Lambda}^{(k)}$, for $k = 1, \dots, \nu$ increases until $\textnormal{dim}~\bm{\Lambda}^{(\nu)} = n$ and the matrix $\begin{bmatrix}\bm{E}^{(\nu)} & \bm{A}^{(\nu)}\end{bmatrix}$ coincide with $ \begin{bmatrix}\bm{\Lambda}^{(\nu)} & \bm{A}^{(\nu)}\end{bmatrix}$. At this point the matrix $\bm{\Lambda}^{(\nu)}\in\mathbb{R}^{n\times n}$ is full rank and then the DAE can be easily transformed into an ODE.
		\begin{equation}
			\bm{\Lambda}^{(\nu)}\dot{\bm{x}}^{(\nu)}+ \bm{A}^{(\nu)}\bm{x}^{(\nu)} = \bm{f}^{(\nu)}.
		\end{equation}
		The minimum integer such that $\textnormal{dim}~\bm{\Lambda}^{(\nu)} = n$ is the Kronecker index (or differential index) of the DAE.
		\begin{example}
			Consider the following set of linear differential equation, 
			\begin{subequations}
				\begin{align}
					\dot{x}_{1}(t) + \dot{x}_{2}(t) + \dot{x}_{3}(t) &= 2x_{1}(t) + x_{3}(t) + t, \\
					\dot{x}_{1}(t) + \dot{x}_{2}(t) + \dot{x}_{3}(t) &= x_{1}(t) + x_{2}(t) + t^{3}, \\
					0 &= x_{1}(t) + x_{2}(t) + x_{3}(t) + \sin{t},
				\end{align}
			\end{subequations}
			that we can write in a more compact matrix form as follows,
			\begin{equation}
				\begin{bmatrix}
					1 & 1 & 1 \\
					1 & 1 & 1 \\
					0 & 0 & 0
				\end{bmatrix}
				\begin{bmatrix}
					\dot{x}_{1}(t)\\
					\dot{x}_{2}(t)\\
					\dot{x}_{3}(t)
				\end{bmatrix}
				= 
				\begin{bmatrix}
					2 & 0 & 1\\
					1 & 1 & 0\\
					1 & 1 & 1
				\end{bmatrix}
				\begin{bmatrix}
					x_{1}(t) \\
					x_{2}(t) \\
					x_{3}(t)
				\end{bmatrix}
				+
				\begin{bmatrix}
					t \\
					t^3 \\
					\sin{t}
				\end{bmatrix}.
			\end{equation}
			Which corresponds to the usual form $E\dot{x}(t) = Ax(t) + f(t)$ obvious meaning for the symbols. Now consider to form the matrix $\begin{bmatrix}E & A & f(t) \end{bmatrix}\in\mathbb{R}^{n\times 2 n+1}$ in such a way that, 
			\begin{equation}
				\begin{bmatrix}
					E & A & f(t)
				\end{bmatrix}
				\begin{bmatrix}
					\dot{x}(t) \\
					- x(t) \\
					-1 
				\end{bmatrix}
				= 0
			\end{equation}
			First it is necessary to check if the  pair $(E,A)$ is a regular pencil, 
			\begin{equation}
				\textnormal{det}(A-\lambda_{0}E) = \textnormal{det}~
				\begin{bmatrix}
					2-\lambda_{0} 	&  -\lambda_{0} 	&	  1-\lambda_{0} \\ 
					1-\lambda_{0} 	&  1-\lambda_{0}  	&  -\lambda_{0} \\
					1 				& 1 				& 1
				\end{bmatrix} = 2
			\end{equation}
			the determinant is different from zero, so the pair $(E,A)$ is a regular pencil.
			
			
			Now it is easy to manipulate the following equations
			\begin{equation}
				\begin{bmatrix}
					1 & 1 & 1 & 2 & 0 & 1 & t\\
					1 & 1 & 1 & 1 & 1 & 0 & t^3\\
					0 & 0 & 0 & 1 & 1 & 1 & \sin{t}\\
				\end{bmatrix}
			\end{equation}
		\end{example}
		